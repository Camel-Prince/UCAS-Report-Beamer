% 默认页面大小 4:3
\documentclass[10pt]{ctexbeamer}
% 页面大小 16:10
% \documentclass[10pt, aspectratio=1610]{ctexbeamer}
% 页面大小 16:9
% \documentclass[10pt, aspectratio=169]{ctexbeamer}
% 页面大小 14:9
% \documentclass[10pt, aspectratio=149]{ctexbeamer}
% 页面大小 1.41:1
% \documentclass[10pt, aspectratio=141]{ctexbeamer}
% 页面大小 5:4
% \documentclass[10pt, aspectratio=54]{ctexbeamer}
% 页面大小 3:2
% \documentclass[10pt, aspectratio=32]{ctexbeamer}

\usetheme[logo=UCAS, sublogo=TRUSC]{ucas}
% logo 的选项: CAS, UCAS
% sublogo 的选项: AMSS, AMSS2018, UCAS, TRUSC

% 引入参考文献列表的 .bib 文件, 使用 GB/T 7714-2015 的文献著录规则.
\usepackage[backend=biber, style=gb7714-2015]{biblatex}
\addbibresource{ref.bib}
\usepackage{multicol}
\title[UCAS Beamer (\LaTeX{})]{***报告}
% \subtitle[非官方]{带噪声的社交网络对齐}
\author[G.\,Chen]{\href{mailto:icgw@outlook.com}{
% 王力 2022E8013282167 \\
王子旭 202228013229109}}
\institute[AMSS, CAS]{中国科学院计算技术研究所 \\ 数据智能系统研究中心}
\date[\today]{\today, 中国北京}
\subject{展示主题}
% \keywords{图数据挖掘,异常检测,回顾与总结}

\begin{document}

\newcommand{\SubItem}[1]{
    {\setlength\itemindent{15pt} \item[-] #1}
}

\begin{frame}[plain]
  \maketitle
\end{frame}

\begin{frame}[t]
  \frametitle{目录}
  \begin{multicols}{2}
    \tableofcontents
    \end{multicols}
\end{frame}

\section[第一章]{一、***}\label{sec:1}
\begin{frame}
    \textbf{\LARGE 一、***}
\end{frame}
\subsection{1.1 ***}\label{subsec:1-1}
\begin{frame}
    \frametitle{***}
\end{frame}

\end{document}
